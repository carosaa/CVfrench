\documentclass[letterpaper]{article}
\usepackage{currvita}
\usepackage[french]{babel}
\usepackage[utf8]{inputenc}
\usepackage{graphicx}
%\usepackage[dvips]{epsfig}
\usepackage{charter}
\usepackage[right=3cm,left=3cm,top=3cm,bottom=3cm,headsep=1cm,footskip=1cm]{geometry}
%\parindent      2cm
%\parskip        0.3cm
%\hyphenpenalty=5000
%\tolerance=1000


\pagestyle{empty}
\begin{document}
%%%%%%%%%%%%%%%%%%%%%%
%\mbox{}
%\vskip -4cm
%\hskip -1.5cm
\hfill

%{\parbox{1cm}{\vskip
%24mm\scalebox{.8}{\includegraphics{Carolina_Saavedra.eps}}}

%\includegraphics{Carolina_Saavedra.eps}\\
%
%\begin{figure}
%  % Requires \usepackage{graphicx}
%  \begin{flushright}
%\includegraphics[width=3.5 cm]{fotocur1.png}\\
%  \end{flushright}
%\end{figure}


%%%%%%%%%%%%%%%%%%%%%%%%%%%%%%%%%%%%
%\vskip -40mm %38.1
\setlength{\cvlabelwidth}{45mm}
\begin{cv}{{\huge Carolina SAAVEDRA}\\
carolina.saavedra@loria.fr}

\begin{cvlist}{Informations Personnelles}
\item[Date de naissance] 26 Juin 1982
\item[Nationalité] Chilienne
\item[Lieu de naissance] Punta Arenas, Chili
\item[Situation familiale] Mariée

\item[Adresse professionnelle] Equipe CORTEX, INRIA Nancy-Grand Est,\\
615, rue du Jardin Botanique 54600 Villers-lès-Nancy, France.
\item[Tél. Professionnel] +33 383 592056
\item[Adresse personnelle] 1 rue Durival, 54000 Nancy, France.
\item[Mobile] +33 610 26 84 75
\item[Courrier électronique] \texttt{<carolina.saavedra@loria.fr>}
\item[Sexe] Féminin
%\item[Ocupaci�n] Estudiante Universitaria
\end{cvlist}

\begin{cvlist}{Langues}
\item [Espagnol] Maternelle
\item [Anglais] Upper Intermediate, reading and writing\\
\textbf{[2008]} TOEIC test, Total Score 670
\item [Français] Courant - ``Cours du Soir'' 2009/10 et 2010/11 au Centre d'Accueil et FOrmation Linguistique (CAFOL), Nancy  

\end{cvlist}

\begin{cvlist}{Situation Actuelle}
\item[Depuis Sept. 2009] Doctorante en Informatique\\
  Equipe CORTEX - LORIA/INRIA.\\
  \textbf{Université de Lorraine}\\
  Ecole doctorale IAEM\\
  Sujet: ``Classifieurs de formes temporelles pour une analyse multi-échelles de signaux physiologiques : application aux interfaces cerveau-ordinateurs''\\
  Encadrement: Bernard GIRAU et Laurent BOUGRAIN.
 
\end{cvlist}


\begin{cvlist}{Etudes et formation}
\item[2008] MASTER EN SCIENCES DE L'INGÉNIERIE INFORMATIQUE\\
	  INGÉNIEUR CIVIL INFORMATICIEN\\
        \textbf{Université technique Federico Santa María}\\
	Valparaíso, Chili.

\item[2005] LICENCE EN SCIENCES DE L'INGÉNIERIE
INFORMATIQUE\\
        \textbf{Université technique Federico Santa María}\\
	Valparaíso, Chili


\item[Ecoles d'été]

\textbf{[2011]} NEUROCOMP: Les Interfaces cerveau-Ordinateur, INRIA. Nancy, France.

\textbf{[2009]} Analysis and Models in Neurophysiology, Bernstein Center. Freiburg, Allemagne.

% \item[Field of Interest]
% Intelligent Data Analysis, Pattern Recognition, Computational
% Intelligence, Artificial Intelligence, Machine Learning, Image
% Processing, Robustness, Programming, Biomedical.
% 

\item[Bourses]

\textbf{[2009-2012]} Bourse du Gouvernement français pour effectuer des études doctorales en Informatique,
 Service de Coopération Scientifique et Technique auprès de l’Ambassade de France au Chili et CONICYT.


%\textbf{[2007-2008]} Initiation scientifique, subvention de l'école doctorale de l'université technique Federico Santa María.

\textbf{[2006-2008]} Bourse d'études Master en Sciences de l'ingénierie Informatique à l'Université technique Federico Santa María.

%\textbf{[2006]} Scholarship Attendance for congresses abroad. XI
% Iberoamerican Congress on Pattern Recognition, Cancún, Mexico.
 
% \textbf{[2005]} Scholarship Attendance for congresses abroad. Fourth
% Mexican International Conference on Artificial Intelligence.

\end{cvlist}


\begin{cvlist}{Expérience}

\item[2011- 2013] \textbf{Assistante de Recherche}\\
 INRIA associate team CORTINA\\
Cortex Project team et NEUROMATHCOMP\\
Sujet: CORtex and reTINA modeling from an engineering and computational perspective

\item[2010] \textbf{Relecture}\\
  4th International Symposium on Bio-Medical Informatics and Cybernetics: 
  BMIC 2010. Orlando, Florida, USA.

\item[2009- 2010] \textbf{Assistante de Recherche}\\
International cooperation research STIC-AMSUD 09STIC01. \\
CORTEX project team, INRIA et {\it Département d'Informatique},
  Université technique Federico Santa María.\\
Sujet: Robust single-trial evoked potential detection for brain-computer interfaces using computational intelligence techniques

\item[Mar 2008- Mar 2009] \textbf{Chercheur}\\
  FONDECYT Projet 1070220.\\
{\it Département d'Informatique},
  Université technique Federico Santa María.\\
  Sujet : Ensemble learning strategies for high-dimensional and non-stationary data
  

\item[Déc 2004- 2006] \textbf{Assistante de Recherche}\\
   DGIP-UTFSM Projet 240425.\\
  {\it Département d'Informatique}, Université technique Federico Santa María.

\item[Déc 2004- Jan 2005] \textbf{Stagiaire}\\
{\it Laboratoire INCA},   Université technique Federico Santa María\\
Sujet du stage de recherche : K-Dynamical Self Organizing Maps


\end{cvlist}

\begin{cvlist}{Enseignement}
 \item La totalité de mes enseignement a été donnée en qualité de vacataire. Le tableau suivant 
résume les différents enseignements dont j'ai eu la charge. Je suis également disposé à enseigner en anglais si besoin.

\begin{table}[h!]
{\small
\hspace{-1.5cm}\begin{tabular}{|l|l|l|l|c|}
\hline
\begin{tabular}{l}{\em Ann{\'e}e}\end{tabular} &\begin{tabular}{l}{\em mati{\`e}re}\end{tabular} & \begin{tabular}{l}{\em public}\end{tabular} & \begin{tabular}{l}{\em type d'enseignement}\end{tabular} & \begin{tabular}{c}{\em volume}\\{\em ({\'e}q. TD)}\end{tabular}\\
\hline
\hline
\begin{tabular}{l}2008\end{tabular} & \begin{tabular}{l}{\em Programmation}\\{\em  C et Pascal}\end{tabular} & \begin{tabular}{l} $1^{\mbox{\scriptsize ère}}$ année de Licence  \\
Université technique Federico Santa María\end{tabular}&\begin{tabular}{l}cours magistraux \end{tabular} & 180 h \\
\hline
\begin{tabular}{l}2007\end{tabular} &\begin{tabular}{l}{\em Programmation}\\{\em Pascal et Delphi}\end{tabular} &\begin{tabular}{l} $1^{\mbox{\scriptsize ère}}$ année de Licence\\ Université de Valparaiso\end{tabular} &\begin{tabular}{l} cours magistraux \end{tabular} & 180
h\\
\hline
\begin{tabular}{l}2006\end{tabular} & \begin{tabular}{l}{\em Programmation}\\{\em  C et Pascal}\end{tabular} & \begin{tabular}{l} $1^{\mbox{\scriptsize ère}}$ année de Licence  \\
Université technique Federico Santa María\end{tabular}&\begin{tabular}{l}cours magistraux \end{tabular} & 180 h \\
\hline
\begin{tabular}{l}2004\\2005\end{tabular} &\begin{tabular}{l}{\em Statistiques Informatiques}\end{tabular} &
 \begin{tabular}{l}$4^{\mbox{\scriptsize {\`e}me}}$ ann{\'e}e d'informatique\\ Université technique Federico Santa María\end{tabular} & \begin{tabular}{l}travaux pratiques\end{tabular} & 60 h \\
\hline
\end{tabular}

}
\caption{R{\'e}sum{\'e} des enseignements}
\end{table}

\end{cvlist}
\newpage
\begin{cvlist}{Compétences}
 \item[Systèmes d'exploitation] Linux, MS-DOS, Windows
\item[Langages de programmation] Matlab, C, C++, Visual Basic, Java, Pascal, Delphi, SQL, UML.  
\item[Bureautique] Latex, Word, Excel, Open Office, Power Point.
\end{cvlist}


% \begin{cvlist}{Avancement de thèse}
%  \item Je suis actuellement en troisième année de doctorat à l'université de Lorraine dans l'équipe Cortex du LORIA. 
% Mon sujet de thèse porte sur l'analyse de signaux cérébraux dans le cadre des interfaces 
% cerveau-ordinateur. Les interfaces cerveau-ordinateur permettent de 
% contrôler un système à partir de l'activité cérébrale. Ces dispositifs sont difficiles 
% à fiabiliser car i) un fort bruit est présent dans les signaux électroencé- phalographiques, ii) il y a de fortes 
% variances intra et inter-individuelles. Il est donc nécessaire de développer de nouvelles techniques de pré-traitement pour les signaux cérébraux et des méthodes de reconnaissance de formes temporelles robustes. Au cours de ma thèse, je me suis intéressée plus particulièrement 
% aux ondelettes afin de proposer une technique pour mesurer la similarité entre des signaux. Mes travaux ont contribué à des projets nationaux et internationaux (tri de potentiels d'action pour le projet ANR Keops, comparaison de méthodes de pré-traitement pour le projet STIC-AmSud BCI). J'ai débuté la rédaction de mon manuscrit de thèse avec pour objectif de soutenir l'année prochaine.
% \end{cvlist}


\begin{cvlist}{Publications}

\item[Revue]

\textbf{[2011]} R. Salas, \textbf{C. Saavedra}, H. Allende and C. Moraga.\\
{\it ``Machine Fusion to enhance the topology preservation of vector quantization artificial neural networks.''}\\
Pattern Recognition Letters, Volume 32(7): 962-972 (2011).

\item[Conférences internationales]

\textbf{[2013]} C. Saavedra and L. Bougrain.\\
{\it ``Denoising and Time-window selection using Wavelet-based Semblance for improving ERP
detection''}\\
International BCI meeting,Pacific Grove, California, 2013.

\textbf{[2013]} C. Saavedra abd L. Bougrain.\\
{\it ``Wavelet-based Semblance for P300 Single-trial Detection''}\\
International Conference on Bio-Inspired Systems and Signal Processing, BIOSIGNAL, 2013.

\textbf{[2012]} L. Bougrain, \textbf{C. Saavedra}, R. Ranta.\\
{\it ``Finally, what is the best filter for P300 detection?''}\\
3rd TOBI Workshop 2012, W\"urzburg, Germany.

\textbf{[2010]} C. Ledesma-Ramirez, E. Bojorges-Valdez, O. Yanez-Suarez,
 \textbf{C. Saavedra}, L. Bougrain, and G. Gentilleti.\\
{\it ``An open-Access P300 Speller Database.''}\\
BCI meeting 2010. Asilomar, California.

% \textbf{[2010]} L. Bougrain, \textbf{C. Saavedra}, B. Payan, A. Hutt, M. Rio and F. Alexandre.\\
% {\it ``INRIA CORTEX team-project: BCI activities.''}\\
% BCI meeting 2010. Asilomar, California.

\textbf{[2009]} \textbf{C. Saavedra}, R. Salas, H. Allende and C. Moraga.\\
{\it ``Fusion of Topology preserving neural Networks.''}\\
Hais 2009 (E. Corchado et al EDS.) LNAI 5572, pp. 517-524, Springer, 2009.

\textbf{[2007]} \textbf{C. Saavedra}, R. Salas, S. Moreno and H. Allende.\\
{\it ``Fusion of Self Organizing Maps.''}\\
Lecture Notes in Computer Science Volume 4507, pp. 227-234. Springer
Verlag.

\textbf{[2007]} S. Moreno, H. Allende, R. Salas and \textbf{C. Saavedra}\\
{\it ``Fusion of Neural Gas.''}\\
Lecture Notes in Computer Science Volume 4527, pp. 558-567. Springer
Verlag.

\textbf{[2006]} \textbf{C. Saavedra}, S. Moreno, R. Salas and H. Allende.\\
{\it ``Robustness Analysis of the Neural Gas Learning Algorithm.''}\\
Lecture Notes in Computer Science Volume 4225, pp. 559-568. Springer
Verlag (ISI).

\textbf{[2005]} R. Salas, S. Moreno, H. Allende and \textbf{C. Saavedra}.\\
{\it ``Flexible Architecture of Self Organizing Maps in changing environmets.''}\\
Lecture Notes in Computer Science Volume 3773, pp. 642-653. Springer
Verlag (ISI).

\textbf{[2005]} \textbf{C. Saavedra}, H. Allende, S. Moreno and R. Salas.\\
{\it ``The K-Dynamical Self Organizing Map.''}\\ Lecture Notes in
Artificial Intelligence. Volume 3789, pp. 702-711. Springer Verlag
(ISI).

\item[Conférence nationale]

\textbf{[2010]} \textbf{C. Saavedra}, L. Bougrain.\\
{\it ``Wavelet Denoising for P300 Single-trial detection.''}\\
Proceedings of 5th french conference in computational neuroscience, Neurocomp, 2010. Lyon, France.
%
% \item[Presentations]
% 
% \textbf{[2007]} 9th International Work-Conference on Artificial
% Neural Networks, June 20-22, San Sebasti�n, Spain. {\it ``Fusion of 
% Self Organizing Maps''}.
% 
% \textbf{[2007]}  International Work-conference on the Interplay
% between Natural and Artificial Computation, June 18-21, La Manga del
% Mar Menor, Murcia, Spain. {\it ``Fusion of Neural Gas''}.
% 
% \textbf{[2006]} XI Iberoamerican Congress on Pattern Recognition,
% November 14-17, Canc�n, Mexico. {\it ``Robustness Analysis of the
% Neural Gas Learning Algorithm''}.
% 
% \textbf{[2005]} Forth Mexican International Conference on Artificial
% Intelligence, 14-18 Noviembre, Monterrey, Mexico. {\it ``The
% K-Dynamical Self Organizing Map''}.
\end{cvlist}


% \begin{cvlist}{Motivacion}
% \item 
%  Les raisons pour lesquelles je souhaite poursuivre le ATER est mon
% désir de poursuivre une carrière universitaire en raison de ma vocation et
% intérêt pour la recherche et l'enseignement. Les dernières années de ma vie ont
% été consacrées à approfondir mes connaissances dans différents domaines de
% recherche, en participant à divers projets, activement publication et de servir
% de professeurs à temps partiel dans la région de Valparaiso.
% Comme le montre dans mon CV et publications développé en collaboration avec mes
% collègues, le principal domaine d'intérêt pour ma recherche est le ``intelligence
% informatique'', mon intérêt principal étant la ``neuro-informatique'' c'est
% pourquoi je suivre me doctorat dans l'institut du groupe CORTEX INRIA, qui
% étudie le fonctionnement du cerveau humain en utilisant des techniques
% d'intelligence computationnelle, en particulier dans les ``interfaces
% cérébrales'', qui a pas de développement en cours au Chili. 
% Je suis convaincu qu'en combinant mes connaissances et
% sa passion pour le enseignement je suis un excellent candidat pour obtenir 
% le ATER, pour offrir de nouvelles connaissances et l'expérience qui sera grande contribution à
% moi comme professionnel.
% 
% \end{cvlist}

 \begin{cvlist}{References}
 \item PhD. Laurent Bougrain.\\
 Phone: (+33) (0)3 83 59 20 54  Mail: \texttt{<laurent.bougrain@loria.fr>}\\
 \emph{Université de Lorraine / LORIA}.

 \item Dr. Ing. Rodrigo Salas F.\\
 Phone: (+56)(32) 2686848 Anexo 603 Mail: \texttt{<rod.salas@gmail.com>}\\
 \emph{Biomedical Department}, Université de Valparaíso.
 
 \item Dr. Ing. Héctor Allende O.\\
 Phone: (+56) (32) 2654313 Mail: \texttt{<hallende@inf.utfsm.cl>}\\
 \emph{Computer Science Department}, Université technique Federico
 Santa Maria.
\end{cvlist}



\end{cv}
\vfill
\hfill Carolina SAAVEDRA
\end{document}
% \item[1999] \textbf{Pialcomp -- Punta Arenas}
%
% Soporte t�cnico de Computadores.
%
% \item[1997 -- 1998] \textbf{Zona Franca -- Punta Arenas}
%
% Vendedora de juguetes en tienda Zona Franca.
